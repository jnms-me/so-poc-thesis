\documentclass[a4paper,12pt]{report}
\usepackage[top=1in, bottom=1in, left=1in, right=1in]{geometry}
\usepackage[dutch]{babel}
\usepackage{graphicx}
\graphicspath{{./img/}}
\usepackage[hidelinks]{hyperref}
\usepackage{pdfpages}

\usepackage{blindtext}

\usepackage{tocbibind}

\usepackage[T1]{fontenc}

\usepackage{listings}
\lstdefinestyle{code}{
    basicstyle=\ttfamily\footnotesize,
    breakatwhitespace=false,         
    breaklines=true,                 
    captionpos=b,                    
    keepspaces=true,                 
    numbers=left,                    
    numbersep=5pt,                  
    showspaces=false,                
    showstringspaces=false,
    showtabs=false,                  
    tabsize=2
}
\lstset{style=code}

\usepackage[backend=biber,style=apa]{biblatex}
\addbibresource{references.bib}

\title{Implementatie van een NIDS binnen een ISP omgeving}
\author{Jonas Meeuws}

\begin{document}

% Bedrukte kaft (=titelblad)
\maketitle
%\includepdf[pages=-]{./include/titelblad.pdf}

% Schutblad
\newpage
\thispagestyle{empty}
\mbox{}

% Titelblad
\maketitle
%\includepdf[pages=-]{./include/titelblad.pdf}

% Mededeling
\chapter*{Mededeling}
\addcontentsline{toc}{chapter}{Mededeling}
Deze eindverhandeling was een examen.
De tijdens de verdediging geformuleerde opmerkingen werden niet opgenomen.

% Woord vooraf (niet verplicht)
\chapter*{Woord vooraf}
\addcontentsline{toc}{chapter}{Woord vooraf}
Hier heb je de mogelijkheid om je waardering uit te drukken en dank te betuigen. Je neemt in
het woord vooraf geen gegevens op die essentieel verband houden met het behandelde
onderwerp (methode, doelstelling, enz.).

% Samenvatting of abstract
\chapter*{Samenvatting}
\addcontentsline{toc}{chapter}{Samenvatting}
\blindtext

% Inhoudsopgave
\tableofcontents
\newpage

% Lijst van figuren (niet verplicht)
\listoffigures

% Inleiding
\chapter*{Inleiding}
\addcontentsline{toc}{chapter}{Inleiding}
\blindtext

% Inhoud
\chapter{Intrusion detection}
Intrusion detection is het monitoren van systemen op verdachte activiteit.
Dit omvat pogingen tot hacken, verspreiding van virussen of malware, aanwezigheid van rogue devices\dots
\\
Intrusion detection gebeurt typisch volledig passief.
De logs die een IDS (Intrusion detection system) produceert kunnen worden verstuurd naar een administrator als een alert.
In grotere systemen worden deze logs centraal verzameld om deze anomalieën effectief te kunnen analyseren.
\autocite{wikipedia:ids}
\\
Intrusion detection systemen worden opgedeeld in 2 soorten: network (NIDS) en host (HIDS) intrusion detection.

\section{NIDS}
Een network intrusion detection systeem (NIDS) monitort een netwerkverbinding.
\\
De 2 grote NIDS software projecten zijn Snort en Suricata.
Voor een groot deel werken de 2 systemen identiek voor een eindgebruiker.
Zo is de NIDS rule syntax van Snort en Suricata bijna gelijk.
Het verschil ligt in specifieke features (zie \url{https://suricata.readthedocs.io/en/suricata-5.0.3/rules/differences-from-snort.html} voor een gedetailleerde lijst van verschillen).

\subsection{Regels}
Een NIDS monitort gecapteerd netwerkverkeer aan de hand van een lijst van regels.
Deze regels volgen een bepaalde syntax.
\\
Een voorbeeld van een nids rule is als volgt (figuur \ref{fig:nids-rule}).
\begin{figure}[h]
  \begin{lstlisting}
alert tcp $EXTERNAL_NET any -> $INTERNAL_NET 3306 (msg:"Inbound SQL connection"; sid:1000000;)
  \end{lstlisting}
  \caption{Een nids rule.}
  \label{fig:nids-rule}
\end{figure}
Deze rule zorgt ervoor dat er een alert wordt gegenereerd wanneer de NIDS engine een tcp verbinding ziet van het extern netwerk naar het intern netwerk op poort 3306.
Het extern netwerk is meestal het internet of een WAN, het intern netwerk is meestal een LAN.
Volgens het tcp client-server model is de client een externe host en de server een interne host.
\\
Poort 3306 is gereserveerd voor MySQL database verbindingen \autocite{iana:ports}.
Deze verbindingen kunnen perfect normaal zijn als de client een vertrouwde host is.
In dit geval is de client echter een externe host die acties probeert uit te voeren op een interne MySQL database.
Dit is verdachte activiteit, vandaar het type alert.

\subsection{Variabelen}
Om regels zo effectief mogelijk te maken, worden er variabelen geïntroduceerd.
De 2 meest gebruikte variabelen zijn het \verb!INTERNAL_NET! en het \verb!EXTERNAL_NET!.
Bijna alle regels gebruiken minstens 1 van deze variabelen, daarom is het van groot belang om deze in te stellen.
Als een variabele niet ingesteld is, worden regels die deze variabele gebruiken genegeerd.
\\
Variabelen kunnen ook de waarde \verb!any! gebruiken.
Dit kan zorgen voor veel false alerts, maar is wel nuttig in gesloten netwerken waar incidenten zich enkel intern afspelen.

% Subscription models
% strategisch plaatsen
\subsection{Port mirror}
% Limitatie linksnelheid (of andere term)
% Schets hoe het netwerk eruit ziet tov de mirror op laag 2 en op laag 3 voor of na de NAT
Een NIDS engine zoals Snort of Suricata wordt normaal gebruikt om te capteren op 1 netwerkinterface.
Deze interface ontvangt typisch een kopie van een punt in het netwerk.
Dit noemt men een port mirror.
\\
Een voorbeeld van een technologie die port mirroring toelaat is SPAN.
Deze is aanwezig in bijvoorbeeld een switch en kopieert alle pakketten van een source poort naar een destination poort.
Daarnaast kan RSPAN gebruikt worden om gecapteerde pakketten over een laag 2 netwerk te transporteren.
Ten slotte kan ERSPAN op de zelfde manier gebruikt worden om een port mirror te transporteren over een laag 3 netwerk.
\autocite{cisco:span}
\\
Het geheel van een NIDS engine die al het netwerkverkeer op 1 punt in een netwerk capteert, noemt men een sensor.
Eén of meerdere sensoren kunnen in een netwerk geplaatst worden op strategische punten.
Dit zijn vaak plaatsen waar een vertrouwd netwerk wordt verbonden met een extern netwerk.
\\
De meest interresante plaats om een sensor te plaatsen in een bedrijfsnetwerk is meestal de plaats waar NAT toegepast wordt.
Dit is omdat alle communicatie met externe netwerken doorheen dit punt passeert.
Als zo'n sensor aan de buitenkant van de NAT staat, zijn de interne ip-adressen niet zichtbaar.
We zien enkel ip-communicatie tussen de NAT-pool en externe netwerken, zo gaat er veel informatie verloren.
Daarom gaat de voorkeur naar het plaatsen van de sensor aan de binnenzijde van de NAT.
\\
Waar men ook voor moet uitkijken bij het plaatsen van een sensor is of de hardware de bitsnelheid wel aankan.
Als de link tussen de port mirror en de sensor een lager maximum bits/s heeft dan bitsnelheid van de combinatie van de beide richtingen van de te capteren poort, kan er packet loss optreden.
NIDS engines kunnen hiermee wel overweg, maar het zorgt natuurlijk voor verlies van informatie.

\section{HIDS}
% Omschrijving principe: agents op clients loggen naar een master server
% Niet uitgebreid
% Zelf nooit gebruikt, nog opzoeken

\chapter{Security Onion}
% Korte geschiedenis van de distro
% Huidige staat van het volledige pakket
% Doelstellingen van SO, een tool voor analysts, gericht op enterprise netwerken
% Elk stack (log sources -> logstash -> elasticsearch -> kibana & elastalert)
% Nids stack (suricata -> sguild -> squert)

\chapter{Ontwikkelen van een aangepast systeem}
% Proberen schema hier te verwerken
\section{Gebruikte technologie\"en}
\subsection{Debian}
% Simpelweg een host os nodig
% Korte uitleg fysieke server geconfgureerd en in serverruimte geplaatst
\subsection{Docker}
% Docker presenteren als tool om linux containers of namespaces (een kernel feature) te beheren
% Docker kan een stuk meer zoals met windows containers, maar dit is alles wat we gebruiken
\subsubsection{Docker compose}
% Github repo presenteren
% Indien het overzichtelijk gebruikt wordt, tool om het volledige systeem mee te ontwerpen
\subsection{Libvirt}
% Open source hypervisor
% Werkt samen met qemu en KVM
% Uitleggen wat een hypervisor is en bekende alternatieven zoals hyper-v (windows) en virtualbox aanhalen

\section{Security Onion}
\subsection{Intern netwerk}
% SO en so-poc zitten in een apart /24 subnet
% 1 wordt beheerd door docker, het andere door libvirt
\subsection{Proxy}
% Web proxy wordt ingezet om ze te verbinden
% Https eerst downgraden, binnen docker http, in https-portal weer upgraden

\section{Web app}
\subsection{React}
% Javascript framework
% States beheren enz, uitleg overnemen van docs
% Material ui library
% dir tree, uitleggen hoe project functioneert
% Homepage en query page
\subsection{Vibe.d}
% Korte omschrijving dlang als general purpose compiled lang
% vibe-d als web framework die toe laat om rest servers en clients te maken
% Werking backend

\section{TheHive}
\subsection{Cortex}
\section{MISP}
\section{Ntop}
\section{Netdata}

\section{Passive os fingerprinting}
\subsection{TCP/IP stack}
\subsection{DHCP}

\section{Visualisatie netwerk layout}
% Uitleggen dat dit passief zeer moeilijk is, geen producten beschikbaar
% Enkel onderscheid maken int->ext of ext->int, distance mogelijk bepalen met time to live
% Traceroute is actief, tools als nmap hiervoor gemaakt

\chapter{De workflow van een analyst}

% Latex ergens vermelden, github actions

% Besluit
\chapter*{Besluit}
\addcontentsline{toc}{chapter}{Besluit}
\blindtext

% Bijlagen (niet verplicht)
\chapter*{Bijlagen}
\addcontentsline{toc}{chapter}{Bijlagen}
\blindtext

% Literatuurlijst
\printbibliography
\addcontentsline{toc}{chapter}{Bibliografie}

% Schutblad
\newpage
\thispagestyle{empty}
\mbox{}

% Kaft
\newpage
Kaft

\end{document}