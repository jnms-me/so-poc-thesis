\documentclass[a4paper, 12pt]{report}
\usepackage[top=1in, bottom=1in, left=1in, right=1in]{geometry}
\usepackage[dutch]{babel}
\usepackage{graphicx}
\graphicspath{{./img/}}
\usepackage[hidelinks]{hyperref}
\usepackage{pdfpages}

\usepackage{blindtext}

\usepackage{tocbibind}

\usepackage[backend=biber,style=apa]{biblatex}
\addbibresource{references.bib}

\title{Implementatie van een NIDS binnen een ISP omgeving}
\author{Jonas Meeuws}

\begin{document}

% Bedrukte kaft (=titelblad)
\maketitle
%\includepdf[pages=-]{./include/titelblad.pdf}

% Schutblad
\newpage
\thispagestyle{empty}
\mbox{}

% Titelblad
\maketitle
%\includepdf[pages=-]{./include/titelblad.pdf}

% Mededeling
\chapter*{Mededeling}
\addcontentsline{toc}{chapter}{Mededeling}
Deze eindverhandeling was een examen.
De tijdens de verdediging geformuleerde opmerkingen werden niet opgenomen.

% Woord vooraf (niet verplicht)
\chapter*{Woord vooraf}
\addcontentsline{toc}{chapter}{Woord vooraf}
Hier heb je de mogelijkheid om je waardering uit te drukken en dank te betuigen. Je neemt in
het woord vooraf geen gegevens op die essentieel verband houden met het behandelde
onderwerp (methode, doelstelling, enz.).

% Samenvatting of abstract
\chapter*{Samenvatting}
\addcontentsline{toc}{chapter}{Samenvatting}
\blindtext

% Inhoudsopgave
\tableofcontents
\newpage

% Lijst van figuren (niet verplicht)
\listoffigures

% Inleiding
\chapter*{Inleiding}
\addcontentsline{toc}{chapter}{Inleiding}
\blindtext

% Inhoud
\chapter{Template}
\section{Template}
\subsection{Template}
\subsubsection{Template}
\blindtext
\autocite{example}

\chapter{Intrusion detection}
\section{NIDS}
% Snort, suricata
% Rules, rule syntax
% Subscription models
% Configuratie: HOME_NET, EXTERNAL_NET
\subsection{Port mirror}
% Kort principe van een port mirror
% Limitatie linksnelheid (of andere term)
% Schets hoe het netwerk eruit ziet tov de mirror op laag 2 en op laag 3 voor of na de NAT
% Altijd verbindingen tussen 2 verschillende netwerken tussen 2 routers bvb
\section{HIDS}
% Omschrijving principe: agents op clients loggen naar een master server
% Niet uitgebreid

\chapter{Security Onion}
% Korte geschiedenis van de distro
% Huidige staat van het volledige pakket
% Doelstellingen van SO, een tool voor analysts, gericht op enterprise netwerken

\chapter{Ontwikkelen van een aangepast systeem}
\section{Gebruikte technologie\"en}
\subsection{Debian}
% Simpelweg een host os nodig
% Korte uitleg fysieke server geconfgureerd en in serverruimte geplaatst
\subsection{Docker}
% Docker presenteren als tool om linux containers of namespaces (een kernel feature) te beheren
% Docker kan een stuk meer zoals met windows containers, maar dit is alles wat we gebruiken
\subsubsection{Docker compose}
% Github repo presenteren
% Indien het overzichtelijk gebruikt wordt, tool om het volledige systeem mee te ontwerpen
\subsection{Libvirt}
% Open source hypervisor
% Werkt samen met qemu en KVM
% Uitleggen wat een hypervisor is en bekende alternatieven zoals hyper-v (windows) en virtualbox aanhalen

\section{Security Onion}
\subsection{Intern netwerk}
% SO en so-poc zitten in een apart /24 subnet
% 1 wordt beheerd door docker, het andere door libvirt
% Web proxy wordt ingezet om ze te verbinden
% Https eerst downgraden, binnen docker http, in https-portal weer upgraden

\section{Web app}
\subsection{React}
% Javascript framework
% States beheren enz, uitleg overnemen van docs
% Material ui library
% dir tree, uitleggen hoe project functioneert
\subsection{Vibe.d}
% Korte omschrijving dlang als general purpose compiled lang
% vibe-d als web framework die toe laat om rest servers en clients te maken
% Werking backend

\section{TheHive}
\subsection{Cortex}
\section{MISP}
\section{Ntop}
\section{Netdata}

\section{Passive os fingerprinting}
\subsection{TCP/IP stack}
\subsection{DHCP}

\section{Visualisatie netwerk layout}
% Uitleggen dat dit passief zeer moeilijk is, geen producten beschikbaar
% Enkel onderscheid maken int->ext of ext->int, distance mogelijk bepalen met time to live
% Traceroute is actief, tools als nmap hiervoor gemaakt

\chapter{De workflow van een analyst}

% Latex ergens vermelden, github actions

% Besluit
\chapter*{Besluit}
\addcontentsline{toc}{chapter}{Besluit}
\blindtext

% Bijlagen (niet verplicht)
\chapter*{Bijlagen}
\addcontentsline{toc}{chapter}{Bijlagen}
\blindtext

% Literatuurlijst
\printbibliography
\addcontentsline{toc}{chapter}{Bibliografie}

% Schutblad
\newpage
\thispagestyle{empty}
\mbox{}

% Kaft
\newpage
Kaft

\end{document}